% Problem 1-1

From the program in \path{rs/clrs_compruntime/src/main.rs}, we have the following
output:
\small\begin{verbatim}
----------------------------------------------------------
f(n)            1 s             1m              1h        
----------------------------------------------------------
lg(n)                inf             inf             inf  
sqrt(n)   1.00000000e+12  3.60000000e+15  1.29600000e+19  
n                1000000        60000000  3.60000000e+09  
nlg(n)             62746         2801417       133378058  
n^2                 1000            7745           60000  
n^3                   99             391            1532  
2^n                   19              25              31  
n!                    10              12              13  
----------------------------------------------------------
--------------------------------------------------------------
      1d              1m              1y              1c
--------------------------------------------------------------
           inf             inf             inf             inf
7.46496000e+21  6.71846400e+24  9.94519296e+26  9.94519296e+30
8.64000000e+10  2.59200000e+12  3.15360000e+13  3.15360000e+15
2.75514751e+09  7.18708564e+10  7.97633893e+11  6.86109568e+13
        293938         1609968         5615692        56156922
          4420           13736           31593          146645
            36              41              44              51
            14              16              17              18
--------------------------------------------------------------
\end{verbatim}\normalsize
\qquad In generating the above results, we have made use of the following two Rust 
functions in \path{rs/clrs_big_oh/src/lib.rs}:
\small\begin{verbatim}
pub fn inverse_nlogn(x: f64) -> f64 {
    let max_iters = 10;
    let mut a_0 = x / (x.log2());
    let mut a_1: f64 = 0.0;

    for _ in 0..max_iters {
        a_1 = a_0 - (a_0 * a_0.log2() - x) / ((1.0 / LN_2) + a_0.log2());
        if ((a_1 * a_1.log2()) - (a_0 * a_0.log2())).abs() < 1.0 {
            return a_1.floor();
        } else {
            a_0 = a_1;
        }
    }

    a_1.floor()
}
\end{verbatim}\normalsize
and:
\small\begin{verbatim}
pub fn inverse_factorial(x: f64) -> f64 {
    let mut i = 0.0;
    let mut fact = 1.0;

    while fact < x {
        if i > 0.0 {
            fact *= i;
        }
        i += 1.0;
    }

    i - 1.0
}
\end{verbatim}\normalsize
\texttt{inverse\_nlogn} makes use of Newton's method to calculate an approximate 
value $m(x)$ s.t. 
\[
    n\lg{n} = x \Rightarrow m(x) \approx n, 
\]
while \texttt{inverse\_factorial} generates an approximate value $m(x)$ s.t. 
\[
    x = n! \Rightarrow m(x) = \lfloor n \rfloor. 
\]
