% Problem 1-1

From the program \path{src/algorithms/big_oh/comp_runtime.cpp}, we have the following
output:

\begin{adjustbox}{width={\textwidth},totalheight={\textheight},keepaspectratio}
    \begin{tabular}{@{} l | r | r | r | r | r | r | r |}
        \qquad & \multicolumn{1}{c |}{1 second} & \multicolumn{1}{c |}{1 minute} 
               & \multicolumn{1}{c |}{1 hour} & \multicolumn{1}{c |}{1 day} 
               & \multicolumn{1}{c |}{1 month} & \multicolumn{1}{c |}{1 year} 
               & \multicolumn{1}{c |}{1 century} \\
        \hline
        $\lg\,n$  & \texttt{2e+06}  & \texttt{1.15292e+24} & \texttt{inf}
        & \texttt{inf}         & \texttt{inf}         & \texttt{inf}         & \texttt{inf} \\
        \hline
        $\sqrt{n}$      & \texttt{1e+12}  & \texttt{3.6e+15}     & \texttt{1.296e+19}
        & \texttt{7.46496e+21} & \texttt{3.65783e+23} & \texttt{9.95841e+26} & \texttt{9.95841e+30} \\
        \hline
        $n$             & \texttt{1e+06}  & \texttt{6e+07}       & \texttt{3.6e+09}
        & \texttt{8.64e+10}    & \texttt{6.048e+11}   & \texttt{3.1557e+13}  & \texttt{3.1557e+15} \\
        \hline
        $n\lg\,n$ & \texttt{62746}  & \texttt{2.80142e+06} & \texttt{1.33378e+08}
        & \texttt{2.75515e+09} & \texttt{1.77631e+10} & \texttt{7.98145e+11} & \texttt{6.86552e+13} \\
        \hline
        $n^2$           & \texttt{1000}   & \texttt{7745}        & \texttt{60000}
        & \texttt{293938}      & \texttt{777688}      & \texttt{5.61756e+06} & \texttt{5.61756e+07} \\
        \hline
        $n^3$           & \texttt{99}     & \texttt{391}         & \texttt{1532}
        & \texttt{4420}        & \texttt{8456}        & \texttt{31600}       & \texttt{146678} \\
        \hline
        $2^n$           & \texttt{19}     & \texttt{25}          & \texttt{31}
        & \texttt{36}          & \texttt{39}          & \texttt{44}          & \texttt{51} \\
        \hline
        $n!$            & \texttt{10}     & \texttt{12}          & \texttt{13}
        & \texttt{14}          & \texttt{15}          & \texttt{17}          & \texttt{18} \\
        \hline
    \end{tabular}
\end{adjustbox}
\qquad In generating the above table, we have made use of the following two C++ functions:
\begin{verbatim}
double inverse_nlogn(double x) {
  uint8_t max_iters = 10;
  double a_0 = x / std::log2(x), a_1 = 0.0;

  for (uint8_t i = 0; i < max_iters; ++i) {
    a_1 = a_0 - ((a_0 * std::log2(a_0) - x) /
                 ((1.0 / std::log(2.0)) + std::log2(a_0)));
    if (abs((a_1 * std::log2(a_1)) - (a_0 * std::log2(a_0))) < 1)
      return a_1;
    else
      a_0 = a_1;
  }

  return a_1;
}
\end{verbatim}
and 
\begin{verbatim}
double inverse_factorial(double x) {
  double i = 0, fact = 1;

  while (fact < x) {
    if (i > 0)
      fact *= i;
    i++;
  }

  return i - 1;
}
    
\end{verbatim}
\texttt{inverse\_nlogn} makes use of Newton's method to calculate an approximate 
value $m(x)$ s.t. 
\[
    n\lg{n} = x \Rightarrow m(x) \approx n, 
\]
while \texttt{inverse\_factorial} generates an approximate value $m(x)$ s.t. 
\[
    x = n! \Rightarrow m(x) = \lfloor n \rfloor. 
\]
