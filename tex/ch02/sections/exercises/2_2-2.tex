% Exercise 2.2-2

\begin{algorithmic}[1]
    \TITLE{\textsc{Selection-Sort}$(A)$}
    \FOR{$i = 1$ \TO $A.\mathit{length} - 1$}
        \STATE $\mathit{smallest} = i$
        \FOR{$j = i$ \TO $A.\mathit{length}$}
            \IF{$A[j] < A[\mathit{smallest}]$}
                \STATE $\mathit{smallest} = j$
            \ENDIF
        \ENDFOR
        \STATE $A[\mathit{smallest}] = A[i]$
    \ENDFOR
\end{algorithmic}

The \textbf{for} loop from lines 2-8 maintains the loop invariant that $A[1\, .\, .\, i-1]$
contains sorted elements in ascending order. 

The algorithm only needs to run for the the first $n - 1$ elements since the $n$th 
element will already be the largest element in the array after each smaller element 
in $A[1\, .\, .\, n]$ has been sorted.

Since there is no \textbf{while} loop and each search for the smallest item in the subarray
$A[i + 1\, .\, .\, n - 1]$ must occur, the best and wors-case running times are the same,
i.e., $\Theta(n^2)$.

The implementation can be seen in the following file:

\path{src/algorithms/sorting/selection_sort.h}
