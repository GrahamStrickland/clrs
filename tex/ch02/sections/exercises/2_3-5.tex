% Exercise 2.3-5

\begin{algorithmic}[1]
  \TITLE{\textsc{Binary-Search}$(A, \nu)$}
  \STATE $\mathit{low} = 1$
  \STATE $\mathit{high} = A.\mathit{length}$
  \WHILE{$\mathit{low} < \mathit{high}$}
  \STATE $\mathit{mid} = \floor{\frac{\mathit{high} + \mathit{low}}{2}}$
  \IF{$A[\mathit{mid}] == \nu$}
  \RETURN $\mathit{mid}$
  \ELSIF{$A[\mathit{mid}] > \nu$}
  \STATE $\mathit{high} = \mathit{mid} - 1$
  \ELSE
  \STATE $\mathit{low} = \mathit{mid} + 1$
  \ENDIF
  \ENDWHILE

  \RETURN $-1$
\end{algorithmic}

Suppose we have an array $A$ of length $n$. Then, in the worst case,
we loop through
the \textbf{while} loop of \textsc{Binary-Search} (lines 3-9) until
$\mathit{low} = \mathit{high}$ and return -1, so that $\nu$ is not in $A$. If we
continually divide $n$ by 2, and $n$ is a power of 2, then we will execute
\[
  \floor{\frac{\mathit{high} - \mathit{low}}{2}}
\]
exactly $\lg{n}$ times. If $n$ is not a power of 2, then the same operation
executes $\floor{\lg{n}}$ times. In any case, we have a worst-case
running time of
$\Theta(\lg{n})$ for \textsc{Binary-Search}.

The implementation can be seen in the following file:

\path{rs/clrs_algorithms/src/search.rs}.
