% Exercise 2.1-3

\begin{algorithmic}[1]
    \TITLE{\textsc{Linear-Search}$(A, \nu)$}
    \STATE $j = 1$
    \WHILE{$j \neq A.\mathit{length}$}
        \IF{$A[j] == \nu$}
            \RETURN $j$
        \ELSE
            \STATE $j = j + 1$
        \ENDIF
    \ENDWHILE
    \RETURN \textrm{NIL}
\end{algorithmic}

\begin{description}
    \item[Initialization:] With the loop invariant being that $A[j]$ refers to an 
        element of the sequence $A = \langle a_1, a_2, \ldots, a_n \rangle$ and 
        each element in $A[1\ltwodots j - 1]$ has been checked for equality, we 
        have the loop invariant valid, since $j = 1$ and $A[1]$ is in $A$ for 
        $n \ge 1$, at the start of the \textbf{while} loop 1 -- 6.
    \item[Maintenance:] Either the \textbf{if} statement in line 3 returns $A[j]$
        or the \textbf{else} statement in line 5 increments $j$ by 1, so that after
        each pass of the \textbf{while} loop, we have checked whether or not 
        $A[j] = \nu$, and the condition of the \textbf{while} loop ensures that 
        $A[j]$ is in $A$.
    \item[Termination:] If the \textbf{while} loop terminates, then either the 
        \textbf{if} condition in line 3 is true, so that $A[j] = \nu$ or 
        $j = A.\mathit{length}$, at which point we return the special value
        \textrm{NIL}.
\end{description}

Thus the algorithm is correct, since the loop invariant is initialized and 
maintained throughout, and the algorithm terminates with the correct output;
if a value is returned, $\nu$ is in $A$, and occurs at $A[j]$ for return value
$j$, otherwise the special value \textrm{NIL} was returned, and $\nu$ is not in $A$.

The implementation can be seen in the following file:

\path{rs/clrs_algorithms/src/search.rs}.
