% Exercise 3.1-6

\begin{proof}
    \begin{itemize}
        \item[$(\Rightarrow)$] Suppose the running time of an algorithm is 
            $\Theta(g(n))$, i.e., $0 \leq c_1 g(n) \leq f(n) \leq c_2 g(n)$ for all 
            $n \geq n_0$, where $c_1, c_2 \in \mathbb{R}^+, n, n_0 \in \mathbb{N}$ and 
            $f(n)$ is a function that describes the running time of the algorithm 
            in question. 

            Now, clearly the worst-case running time of the algorithm is some value 
            $M \leq c_2 g(n)$, since $x = f(n)$ for the greatest possible value of $M$. 
            But then we have $0 \leq x \leq c_2 g(n)$ and the running time $M = O(g(n))$. 

            Likewise, if $x'$ is the best-case running time and $m$ is the minimum value 
            of $f(n)$ for all $n \in \mathbb{N}$ and $0 \leq c_1 g(n) \leq m$, so 
            $m = \Omega(g(n))$.
        \item[$(\Leftarrow)$] Now, suppose the worst-case running time $M = O(g(n))$ and
            the best-case running time $m = \Omega(g(n))$. Then we have
            $0 \leq c_1 g(n) \leq m$ and $0 \leq M \leq c_2 g(n)$ for all $n \geq n_0$,
            where $c_1, c_2 \in \mathbb{R}^+, n, n_0 \in \mathbb{N}$. But clearly
            $m \leq f(n) \leq M$ for all $n$, so that it follows that
            \[
                0 \leq c_1 g(n) \leq m \leq g(n) \leq M \leq c_2 g(n)
            \]
            for all $n > n_0$, i.e., $f(n) = \Theta(g(n))$. 
    \end{itemize} 
\end{proof}
