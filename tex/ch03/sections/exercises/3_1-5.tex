% Exercise 3.1-5

\begin{proof}
  \begin{itemize}
    \item[$(\Rightarrow)$] Suppose $f(n) = \Theta(g(n))$, then we have
      $0 \leq c_1 g(n) \leq f(n) \leq c_2 g(n)$ for all $n \geq n_0$,
      where $c_1, c_2 \in \mathbb{R}^+, n, n_0 \in \mathbb{N}$.

      But then it follows that $0 \leq f(n) \leq c_2 g(n)$ for all $n \geq n_0$,
      so $f(n) = O(g(n))$, and $0 \leq c_1 g(n) \leq f(n)$ for all $n \geq n_0$,
      so $f(n) = \Omega(g(n))$.

    \item[$(\Leftarrow)$]
      Now, suppose $f(n) = O(g(n))$, i.e., $0 \leq f(n) \leq c_2 g(n)$ for all
      $n \geq n_0$ and $f(n) = \Omega(g(n))$, i.e., $0 \leq c_1 g(n) \leq f(n)$
      for all $n \geq n_0$, where $c_1, c_2 \in \mathbb{R}^+, n, n_0
      \in \mathbb{N}$.
      But then we have $0 \leq c_1 g(n) \leq f(n) \leq c_2 g(n)$ for
      all $n \geq n_0$,
      so that $f(n) = \Theta(g(n))$.
  \end{itemize}
\end{proof}
