% Exercise A.1-6 

\begin{proof}
    We have
    \begin{equation*}
        \begin{split}
            \sum_{k = 1}^n O(f_k(i)) = \, &O(f_1(i)) + O(f_2(i)) + \cdots + O(f_n(i)) \\
                                     = \, &\lbrace f(i) : \text{ there exist positive constants }
                                     c \text{ and } i_0 \\
                                        &\qquad \text{ such that } 0 \leq f(i) \leq cf_1(i)
                                         \text{ for all } i \geq i_0 \rbrace \\
                                        &+ \lbrace f(j) : \text{ there exist positive constants }
                                     d \text{ and } j_0 \\
                                        &\qquad \text{ such that } 0 \leq f(j) \leq df_2(j)
                                         \text{ for all } j \geq j_0 \rbrace \\
                                        &+ \cdots + \lbrace f(w) : \text{ there exist positive constants }
                                     z \text{ and } w_0 \\
                                        &\qquad \text{ such that } 0 \leq f(w) \leq zf_n(w)
                                         \text{ for all } w \geq w_0 \rbrace.
        \end{split}
    \end{equation*}

    Interpreting summation in the above as set union, we have a superset $\psi$ where 
    $O(f_1(i)) \subseteq \psi, O(f_2(i)) \subseteq \psi, \ldots O(f_n(i)) \subseteq \psi$,
    since $\psi = \sum_{k = 1}^n O(f_k(i))$.

    Now, by the linearity property, $f_1(i) + f_2(i) + \cdots f_n(i) = \sum_{k = 1}^n f_k(i)$
    is in $\psi$, so it follows that $\psi = O\bigl(\sum_{k = 1}^n f_k(i)\bigr)$.
\end{proof}
